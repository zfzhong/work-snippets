\documentclass[11pt, oneside]{book}   	% use "amsart" instead of "article" for AMSLaTeX format
%\documentstyle[fancychapters]{report}
\usepackage{geometry}                		% See geometry.pdf to learn the layout options. There are lots.
\geometry{letterpaper}                   		% ... or a4paper or a5paper or ... 
%\geometry{landscape}                		% Activate for rotated page geometry
%\usepackage[parfill]{parskip}    		% Activate to begin paragraphs with an empty line rather than an indent
\usepackage{graphicx}				% Use pdf, png, jpg, or eps§ with pdflatex; use eps in DVI mode
								% TeX will automatically convert eps --> pdf in pdflatex		
\usepackage{amssymb}
\usepackage{dirtytalk}
\usepackage{epigraph}
\usepackage{csquotes}
\usepackage{quotchap}
\usepackage{soul}

%SetFonts

%SetFonts



\title{Dave's Random Thoughts}
\author{Teacher David}
\date{December 2, 2023}							% Activate to display a given date or no date

\begin{document}
\maketitle
%\section{}
%\subsection{}

\chapter{Dreaming, Intelligence, and Generating Math Problems}
\begin{quotation}
  \emph{In dreams you see reality, sometimes.}
\end{quotation}

Upon awakening yesterday, I found myself contemplating the intricacies
of my brain's functioning. 

In my dream, I encountered individuals in my hometown preparing for a
fishing trip. Eager to join them, I expressed my interest, and they
agreed, proposing that I procure a large umbrella for sun protection
during our fishing expedition. I gladly accepted, recognizing the
minimal cost involved and the valuable experience of fishing with
them. 

Upon waking, I realized that the people from my dream were not actual
fishing enthusiasts in reality. However, the scenario of being asked
to buy an umbrella in exchange for a fishing opportunity mirrors
real-life occurrences. It appears that my brain crafted a narrative
about a fishing excursion with a group and transposed it onto
individuals who don't actually engage in fishing. This projection
maintains logical coherence in its central plot (buying an umbrella
for a fishing opportunity), despite the incongruity in the details
(these individuals don't fish in real life). 

It seems that the brain retains the underlying logic of a story and
later projects that logic onto a group of characters to generate
tangible scenes. This underlying logic serves as a summary of features
and logics from real-life experiences. It is a natural inference that
intelligence involves summarizing and extracting logic and features
from a broad spectrum of data. 

Transitioning to the Kaleidoscope project I am currently engaged in,
my goal is to create a generic model for a math problem and
subsequently generate various math problems based on the same model
but with different parameters. The challenge lies in defining this
generic model. Currently, I employ a table (or matrix) and projection
functions to construct the model, and this approach functions
adequately. 

The crux of the matter is how to autonomously and efficiently
summarize a math problem into this generic model with minimal manual
intervention. If we can achieve the autonomous summarization of a math
problem into a generic model and project the model onto various math
problems sharing the same underlying logic, we can attain a degree of
intelligence in both math problem creation and solving.

\chapter{From Airballs to Buckets: Shoot a Basketball Right}

Just a short time ago, I was that player who dreaded taking a shot. I’d pull up from mid-range and launch an airball. I’d miss wide, long, short—every way possible. My confidence was low, and I honestly thought I just wasn’t cut out to be a shooter.

But I didn’t want to stay that way. I took time to study, practice, and completely rebuild my shot. I watched how good shooters moved, broke down my bad habits, and started making small changes -- one by one.

And it paid off. Just yesterday, I hit 10 shots in a row from just inside the three-point line. That was unthinkable for me a few weeks ago. In this article, I’m sharing the five key lessons that made the difference for me. If you're struggling with your shot like I was, I hope this helps you turn things around.

\section{Power Comes from Your Legs -- Jump Forward}
Early on, I relied on my arms for all the power. That’s why most of my shots came up short or flew awkwardly.

\textit{The secret: good shooters power their shot with their legs.}

You need to bend your knees and generate lift from your lower body. Even better, jump slightly forward -- not just straight up. It doesn’t have to be exaggerated, just a small forward motion that lets your momentum carry into the shot. Once I focused on this, my shots felt smoother and had way more range.

If you’re shooting airballs, chances are you’re not using your legs enough -- or not jumping at all.

\section{Be Balanced -- Land Like a Shooter}
People always talk about “balance,” but for a long time I didn’t really get it. Then I realized:

\textit{Balance means you should land on the ground steadily, under control, and in good posture.}

Here’s what proper balance looks like:
\begin{enumerate}
\item Both feet land at the same time.
\item Your upper body stays upright.
\item Your eyes remain on the rim.
\item Your shooting arm finishes high, with your hand in a relaxed “gooseneck” form above your nose.
\end{enumerate}

If you're fading sideways, kicking your legs out, or spinning after the shot, you’re not balanced -- and that will mess with your accuracy and consistency.

\section{Release with Index and Middle Fingers -- and Spread Your Hand}

I used to flick the ball with my whole hand. My thumb got involved, and the spin was all over the place. Then I learned:

\textit{The last fingers to touch the ball should be your index and middle fingers.}

But here's something just as important:
\begin{enumerate}
\item To make that work, you have to spread your fingers properly when holding the ball.
\item Keep your fingers wide and extended.
\item Maintain a small space between your palm and the ball.
\item Let your index and middle fingers naturally cover the top of the ball.
\end{enumerate}

This gives you the control you need to put a clean backspin on every shot. Without that finger spread, your release will feel tight and inconsistent.

Here's another subtle but crucial point: 

\textit{Feel the pressure from the ball with your index and middle fingers before your start to jump and swing your arm.}

Notice that you must bend your knees to prepare for your jump before you release the ball. When you bend your knees, your body lowers slightly, and the ball drops a bit in height. At that moment, the ball presses down more firmly on your fingers. That’s when you should feel the weight of the ball resting mainly on your index and middle fingers. These two fingers become the main support for the ball, and you’ll notice they naturally carry more of the pressure.

Then you start to jump and swing your arm to release the ball. Your index finger and middle finger naturally push the ball into the air because they already feel the ball's pressure -- it's natural for them to react with strength. This makes your release smoother and more powerful.

Practice more and you will utilize the two fingers more and more effectively. Then your motion in shooting will reduce to minimal. It will look like you don't use much strength to shoot and you will be shooting will great accuracy. You will shoot the ball in elegance.

\section{Always Shoot from the Same Spot -- Next to Your Right Eyebrow}
Shooting is all about repeatability. If your form changes every time, your results will too. I used to bring the ball up in different places -- sometimes in front of my face, sometimes off to the side. Big mistake.

Now, I always bring the ball up to the same place: right next to the eyebrow of my right eye.

That’s my shooting pocket. From there, I align my elbow under the ball, keep my hand ready to snap, and go into my shot with confidence.

And timing matters too:

Release the ball at the peak of your jump -- or just before you reach it.

Too early and the shot is flat. Too late and it’s weak. Find that sweet spot and stick to it.

\section{Finish Strong -- With a Balanced Follow-Through}
After every shot, I check my form. Did I hold the gooseneck? Did I land upright? Am I balanced?

The shot isn’t over when the ball leaves your hand -- it ends when you land.

Holding your follow-through helps build muscle memory and discipline. It teaches your body to complete the motion correctly every single time. Even pros hold their follow-through for a moment after shooting. It’s not just for style -- it’s part of the process.

\section{Final Thoughts}
If I can improve, so can you. I went from dreading my shot to watching it drop with confidence -- and it didn’t take talent. It took patience, focus, and a willingness to start from the ground up.

Here’s a quick recap of what changed my game:
\begin{enumerate}
\item Power your shot with your legs, and jump slightly forward
\item Stay balanced, and land in control
\item Use only your index and middle fingers, and spread your hand
\item Always shoot from the same pocket -- next to your right eyebrow
\item Finish the shot with discipline, and check your follow-through
\end{enumerate}

Keep practicing. Stay consistent. The shots will start to fall -- and when they do, you’ll remember exactly how you earned them.

\end{document}  
