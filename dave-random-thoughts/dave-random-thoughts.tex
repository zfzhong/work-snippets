\documentclass[11pt, oneside]{book}   	% use "amsart" instead of "article" for AMSLaTeX format
%\documentstyle[fancychapters]{report}
\usepackage{geometry}                		% See geometry.pdf to learn the layout options. There are lots.
\geometry{letterpaper}                   		% ... or a4paper or a5paper or ... 
%\geometry{landscape}                		% Activate for rotated page geometry
%\usepackage[parfill]{parskip}    		% Activate to begin paragraphs with an empty line rather than an indent
\usepackage{graphicx}				% Use pdf, png, jpg, or eps§ with pdflatex; use eps in DVI mode
								% TeX will automatically convert eps --> pdf in pdflatex		
\usepackage{amssymb}
\usepackage{dirtytalk}
\usepackage{epigraph}
\usepackage{csquotes}
\usepackage{quotchap}
\usepackage{soul}

%SetFonts

%SetFonts



\title{Dave's Random Thoughts}
\author{Teacher David}
\date{December 2, 2023}							% Activate to display a given date or no date

\begin{document}
\maketitle
%\section{}
%\subsection{}

\chapter{Dreaming, Intelligence, and Generating Math Problems}
\begin{quotation}
  \emph{In dreams you see reality, sometimes.}
\end{quotation}

Upon awakening yesterday, I found myself contemplating the intricacies
of my brain's functioning. 

In my dream, I encountered individuals in my hometown preparing for a
fishing trip. Eager to join them, I expressed my interest, and they
agreed, proposing that I procure a large umbrella for sun protection
during our fishing expedition. I gladly accepted, recognizing the
minimal cost involved and the valuable experience of fishing with
them. 

Upon waking, I realized that the people from my dream were not actual
fishing enthusiasts in reality. However, the scenario of being asked
to buy an umbrella in exchange for a fishing opportunity mirrors
real-life occurrences. It appears that my brain crafted a narrative
about a fishing excursion with a group and transposed it onto
individuals who don't actually engage in fishing. This projection
maintains logical coherence in its central plot (buying an umbrella
for a fishing opportunity), despite the incongruity in the details
(these individuals don't fish in real life). 

It seems that the brain retains the underlying logic of a story and
later projects that logic onto a group of characters to generate
tangible scenes. This underlying logic serves as a summary of features
and logics from real-life experiences. It is a natural inference that
intelligence involves summarizing and extracting logic and features
from a broad spectrum of data. 

Transitioning to the Kaleidoscope project I am currently engaged in,
my goal is to create a generic model for a math problem and
subsequently generate various math problems based on the same model
but with different parameters. The challenge lies in defining this
generic model. Currently, I employ a table (or matrix) and projection
functions to construct the model, and this approach functions
adequately. 

The crux of the matter is how to autonomously and efficiently
summarize a math problem into this generic model with minimal manual
intervention. If we can achieve the autonomous summarization of a math
problem into a generic model and project the model onto various math
problems sharing the same underlying logic, we can attain a degree of
intelligence in both math problem creation and solving.

\end{document}  
